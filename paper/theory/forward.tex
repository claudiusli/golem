\documentclass[12pt]{article}
\usepackage{amssymb,amsmath,amsthm}

\theoremstyle{plain}
\newtheorem{lemma}{Lemma}
\newtheorem{definition}{Definition}

\begin{document}
\title{Conditioning the Forward Logit Operator Through Task Evaluation}
\author{Bill Cochran}
\maketitle

\begin{abstract}
Language generation can be decomposed into six stages: (1) abstraction of meaning into tokens, 
(2) finite sampling of these tokens, (3) application of this sampling to a neural network to
compute internal activations, (4) projection back into the finite token space, (5)
expansion into the infinite token space, allowing for (6) semantic unit selection. We define 
step (5), the token expansion step, as the \emph{forward logit operator}. In this work, we 
propose a mechanism for conditioning this operator with basic task evaluation.
\end{abstract}


\section{Introduction}
In classical neural networks, so the intuition goes, depth corresponds to abstraction. Each 
successive layer composes lower-order features into increasingly complex representations, 
approximating higher-order logical predicates over the input domain. While this phenomenon 
is widely observed empirically, it remains under-characterized in terms of the projection 
geometry between internal representations and output behavior.

In this work, we examine how these high-level internal representations---what we term 
\emph{concept nodes}---can be harnessed to modulate the output distribution in a structured 
and theoretically principled manner. Rather than treating the final-layer logits as a fixed 
projection from hidden state to token space, we view them as an intermediate surface: a 
mutable space in which meaning, preference, and structure may be reintroduced through targeted transformations.

We propose a framework in which a secondary learner—such as a convolutional network, recurrent 
architecture, or multilayer perceptron—operates on the concept node activations to produce a 
transformation in logit space. This transformation serves as a dynamic filter, applied to the 
base logits either multiplicatively or additively, altering the output distribution in a way 
conditioned on higher-order internal structure.

This conditioning mechanism may be interpreted as a second-order logic step, applied not over 
input tokens but over internal concepts. It augments the forward pass with an auxiliary computation 
whose role is to shape the model’s output trajectory according to semantic content encoded at depth. 
Because this transformation is expressed in logit space, it interfaces naturally with token selection 
and decoding, while remaining disentangled from the core model weights.

We formalize this view by treating the final logit vector as an object subject to transformation: 
not merely the result of a projection, but the input to a secondary operator that encodes 
task-specific logic, temporal preference, or structural bias. This layered approach provides a 
means of modulating language generation without altering the underlying network, enabling a clean 
separation between representation and control.

Finally, this perspective suggests a natural interface for injecting structured feedback into a 
language model: the logit space itself. In conventional usage, a language model must infer, 
\emph{a priori}, whether it is performing well—relying solely on implicit cues from the prompt 
and its own training. There is no inherent mechanism for runtime evaluation or correction of 
its current behavior in light of task success.

By contrast, conditioning the output logits via a transformation grounded in concept space provides 
a direct channel through which external or internal feedback \emph{a posteriori}. Rather than 
backpropagating through the entire model, we can apply lightweight corrections that indicate, 
for instance, that certain tokens are ineffective or undesirable in the current context. 
This suggests that task specialization in language models may be achieved through structured 
conditioning of output logits---mirroring the way human learning refines behavior through iterative 
feedback and instruction. Practice makes perfect.


\section{Related Work}

In~\cite{Cochran2025HierarchicalLogic}, we develop a set of linear tensor operators
that describe the language token generation problem.


\bibliographystyle{plain}
\bibliography{golem}
\end{document}
